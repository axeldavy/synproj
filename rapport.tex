\documentclass[a4paper]{article}
\usepackage[latin1]{inputenc}
\usepackage[T1]{fontenc}
\usepackage[francais]{babel}
\usepackage{verbatim}
\usepackage{hyperref} 

\author{Robin Champenois et Axel Davy}

\title{Petite pr�sentation de notre Model Checker}

\begin{document}

\maketitle

\section{Fonctionnalit�s}

Notre Model Checker Lustre supporte
\begin{itemize}
\item Les nodes
\item Les flottants (mais pas \verb!int_of_float! et \verb!float_of_int!)
\item Les tuples
\end{itemize}
Il utilise la k-induction pour prouver les programmes.

\section{Choix d'impl�mentation et difficult�s rencontr�es}
\subsection{Organisation du code}
Le code � �crire a �t� s�par� en deux �tapes principales :
\begin{itemize}
\item La ``traduction'' (\verb!translater.ml!), qui se charge de parcourir l'arbre de syntaxe abstraite, et de g�n�rer les formules Alt-ergo-zero associ�es au programme ;
\item La ``preuve'' (\verb!prover.ml!), qui utilise l'interface d'alt-ergo-zero pour essayer de d�montrer ce qu'il faut d�montrer. C'est dans cette �tape que l'on entre les hypoth�ses, et que l'on r�alise la k-induction.
\end{itemize}
Le tout est 

\section{Traduction}
L'approche initiale a �t� simple : on traite chaque node successivement, on traduit les expressions r�cursivement en g�n�rant (par effet de bord) les formules n�cessaires, en cr�ant des variables auxiliaires si n�cessaires.

Cette approche un peu na�ve ne fonctionnait en r�alit� que pour le cas d'un node, sans appel � d'autres nodes. En r�alit�, comme il faut des variables diff�rentes pour chaque appel de node, il a fallu traiter les nodes dans l'ordre des appels, le nombre de fois qu'ils sont appel�s (en commen�ant par le node principal).

\section{Preuve}
% TODO

\subsections{Optimisations}
Nous n'avons pas r�alis� d'optimisations � la k-induction. 

Certains optimisations importantes, comme d�crites dans les articles, sont impossible du fait que Aez ne retourne pas de contre-exemples lorsqu'une formule est fausse.
\end{document}
